\documentclass[12pt,letterpaper]{article}
\usepackage[a4paper, top=1.2in, bottom=1.4in, left=1in, right=1in]{geometry}
\usepackage{graphicx} % Required for inserting images
\graphicspath{ {./img/} }
\usepackage[spanish]{babel}
\usepackage{float}
\usepackage{fancyhdr}
\setlength{\parskip}{1em}  % Adds space between paragraphs (1em)
\usepackage{amsmath,amssymb}
\usepackage{tikz}
\newcommand{\tikzmark}[1]{\tikz[baseline,remember picture] \coordinate (#1) {};}
\usetikzlibrary{positioning}
\usetikzlibrary{shadows,arrows.meta} % For adding edges label
\usetikzlibrary{calc}
\usepackage{eso-pic}
\usepackage[backend=biber, defernumbers=true, citestyle=numeric-comp, bibstyle=ieee, sorting=none]{biblatex}
\addbibresource{bibliography/bibliography.bib}
\DeclareBibliographyCategory{cited}
\AtEveryCitekey{\addtocategory{cited}{\thefield{entrykey}}}
% Configurando BibLaTeX
\DefineBibliographyStrings{spanish}{
  url = {URL},
  andothers={et ~al\adddot}
}
\usepackage{listings}
\usepackage{xcolor}
\usepackage{algorithm}
\usepackage{algpseudocode}

\definecolor{codegreen}{rgb}{0,0.6,0}
\definecolor{codegray}{rgb}{0.5,0.5,0.5}
\definecolor{codepurple}{rgb}{0.58,0,0.82}
\definecolor{backcolour}{rgb}{0.95,0.95,0.92}

\lstdefinestyle{mystyle}{
    backgroundcolor=\color{backcolour},   
    commentstyle=\color{codegreen},
    keywordstyle=\color{magenta},
    numberstyle=\tiny\color{codegray},
    stringstyle=\color{codepurple},
    basicstyle=\ttfamily\footnotesize,
    breakatwhitespace=false,         
    breaklines=true,                 
    captionpos=b,                    
    keepspaces=true,                 
    numbers=left,                    
    numbersep=5pt,                  
    showspaces=false,                
    showstringspaces=false,
    showtabs=false,                  
    tabsize=2
}

\lstset{style=mystyle}

\AddToShipoutPictureBG{%
\begin{tikzpicture}[remember picture, overlay]
\node[opacity=.15, inner sep=0pt]
    at(current page.center){\includegraphics[scale=1.5]{img/logo-ugr2.png}};
\end{tikzpicture}%
}

\title{Revisión de los Algoritmos Genéticos}
\author{Miguel García López}
\date{Noviembre 2024}

\pagestyle{fancyplain}
\headheight 35pt
\lhead{Miguel García López}            
\chead{\textbf{\small Revisión de los Algoritmos Genéticos}}
\rhead{Master Ciencia de Datos \\ \today}
\lfoot{\scriptsize\LaTeX}
\cfoot{\small Revisión de los Algoritmos Genéticos}
\rfoot{\small\thepage}
\headsep 1.5em

\author{Miguel García López} % Nombre y apellidos

\date{\normalsize\today} % Incluye la fecha actual

\begin{document}
\begin{titlepage}
\begin{figure}
    \vspace{-1.3cm}
    \begin{center}
        \includegraphics[width=0.75\textwidth]{img/UGR-Logo.png}
    \end{center}
\end{figure}
\vspace{1.3cm}
\centering
\normalfont \normalsize
\textsc{\textbf{Técnicas de Soft Computing para Aprendizaje y optimización. Redes Neuronales y Metaheurísticas, programación evolutiva y bioinspirada 2024-2025} \\ \vspace{.15cm} Master Ciencia de Datos\\ \vspace{.15cm} Universidad de Granada} \\ [25pt] 
    \huge Revisión de los Algoritmos Genéticos

\normalfont \normalsize \vspace{.30cm}
\textsc{Miguel García López}

\end{titlepage}

\tableofcontents
\listoffigures
\newpage

\section{Introducción}
En esta revisión se va a proceder a explicar qué son los \textbf{algoritmos genéticos}, de dónde vienen y cómo surgen a lo largo del tiempo como una solución bio-inspirada para solucionar problemas de optimización con codificación binaria. Se explicará cómo estos algoritmos surgieron y fueron ``evolucionando" con el tiempo, adaptándose a nuevos problemas y creando variaciones del algoritmo original. 

Se recabará información a través del buscador de \textit{Scopus} para realizar un análisis bibliométrico, con el que se analizará la producción científica y académica y se medirá el impacto, la calidad y la evolución de publicaciones, autores, revistas e instituciones en el área específica de los algoritmos genéticos.

Se explicará, además, el funcionamiento básico del algoritmo genético original y el de algunas de sus variantes e hibridaciones. Para ello se explicarán los operadores que lo conforman así como el \textit{workflow} o procesos que sigue el algoritmo y cómo interactúan los operadores entre sí para hacer funcionar al algoritmo de optimización.

\section{Contexto}
Los algoritmos genéticos están inspirados en la selección natural y se emplean tanto en problemas de optimización con restricciones como en aquellos sin ellas. Esta metaheurística modifica de manera repetida una población de soluciones individuales, seleccionando soluciones ``padre" que generarán la siguiente generación de soluciones en la siguiente iteración del algoritmo.

En su forma más básica, un algoritmo genético opera sobre una población de soluciones potenciales a un problema dado. Cada solución potencial, frecuentemente llamada individuo o cromosoma, está representada como una cadena de símbolos, que puede ser binaria, numérica o simbólica. \cite{10.5555/522098}.

\subsection{Orígenes}
Los algoritmos genéticos o \textbf{GAs} adquirieron popularidad en la década de $1970$, especialmente en $1975$ con la publicación del libro de John Holland \cite{Holland:1975}. Este tipo de metaheurísticas se diseñan tomando como inspiración la selección natural. Un conjunto de fenotipos (soluciones) evoluciona a lo largo de generaciones para emular el cruce entre especies, es decir, el cruce de soluciones mediante un intercambio común de cromosomas, lo que da lugar a nuevos individuos con características de ambos padres. A lo largo del tiempo, este tipo de algoritmos fueron incorporando nuevas características y, aunque inicialmente fueron concebidos para resolver problemas discretos, también existen versiones que optimizan problemas continuos \cite{eiben2015}.

\subsection{Definición}
Los algoritmos genéticos definen las soluciones como vectores numéricos, en el caso de un problema binario (\textbf{GAs} originales) la solución es un vector de $0s$ y $1s$, en el caso de un problema sobre un dominio continuo, la solución es un vector de números reales.
Cada individuo $x \in P$ se representa como una cadena de genes:
\[ x = (g_1, g_2, ..., g_n) \]
donde $g_i \in \Sigma$ y $\Sigma$ es el alfabeto de genes que puede ser:
\begin{itemize}
    \item Binario: $\Sigma = \{0,1\}$
    \item Entero: $\Sigma = \mathbb{Z}$
    \item Real: $\Sigma = \mathbb{R}$
\end{itemize}
El algoritmo trata de minimizar o maximizar una función objetivo, esta es conocida como la función \textit{fitness}. En esta se representa una métrica de cómo de bien lo está haciendo el algoritmo (o el error) y se trata de optimizar para guiar la evolución de la población.

Se comienza con una población de soluciones totalmente aleatoria partiendo de una distribución (puede ser normal o uniforme). A partir de ahí se evalúan los individuos y se les proporciona un \textit{score} o evaluación a cada uno. Con esa evaluación se ordenan y se eligen dos padres (o varios, dependiendo de la variante) mediante un proceso de selección, como el método de la ruleta.

Esos padres son cruzados entre sí mediante el operador de cruce para obtener uno o más hijos que compartan características de los padres. Dada una probabilidad defindida como $p$ se le aplicará el operador de mutación a los hijos para favorecer la diversidad y exploración.

A partir de ahí, se introducen los hijos en la población, si se pueden introducir, y se vuelve a iterar. La condición de parada puede ser llegar a un valor de \textit{fitness}, condición de tiempo o condición de iteraciones máximas.

\begin{algorithm}
\caption{Algoritmo Genético con Elitismo}
\begin{algorithmic}[1]
\Require Tamaño de población \( N \), tasa de cruce \( C \), tasa de mutación \( M \), porcentaje de elitismo \( E \), condición de terminación
\Ensure La mejor solución encontrada
\State Inicializar población aleatoria \( P \) con \( N \) individuos
\While{no se cumpla la condición de terminación}
    \State Evaluar fitness de cada individuo en \( P \)
    \State Ordenar \( P \) de mejor a peor según fitness
    \State \( P_{\text{elite}} \gets \) primeros \( E \times N \) individuos de \( P \)
    \State Inicializar \( P_{\text{descendencia}} \gets \emptyset \)
    \For{$i$ desde 1 hasta \( N $}
        \State Seleccionar padres \( p_1, p_2 \) de \( P \) usando selección por ruleta/torneo
        \State \textbf{if} \( \text{rand}() < C \) \textbf{then}
            \State \( h \gets \text{Cruce}(p_1, p_2) \)
        \State \textbf{else}
            \State \( h \gets \) copia de \( p_1 \) o \( p_2 \) (aleatorio)
        \State \textbf{endif}
        \State \textbf{if} \( \text{rand}() < M \) \textbf{then}
            \State \( h \gets \text{Mutación}(h) \)
        \State \textbf{endif}
        \State Agregar \( h \) a \( P_{\text{descendencia}} \)
    \EndFor
    \State Reemplazar los \( E \times N \) peores individuos en \( P_{\text{descendencia}} \) con \( P_{\text{elite}} \)
    \State \( P \gets P_{\text{descendencia}} \)
\EndWhile
\State \Return mejor individuo en \( P \)
\end{algorithmic}
\end{algorithm}

\subsection{Operadores principales}
Se procede a definir los operadores principales de los algoritmos genéticos, al menos los básicos o más usados, ya que cubrir todos los operadores de todas las variaciones es inabarcable.

\subsection{Selección}
La selección se puede llevar a cabo por varios métodos. Uno de los más utilizados es \textbf{selección por torneo} (eq: \ref{eq:tournament}). En la selección por torneo, los individuos compiten entre sí en grupos pequeños. Cada grupo (torneo) tiene varios competidores. El competidor con mejor \textit{fitness} dentro del grupo tiene más probabilidades de ganar. Los ganadores son seleccionados para participar en el proceso de cruce.
El proceso se repite hasta que se completa la \textit{pool} de apareamiento con los ganadores de los torneos \cite{miller_genetic_nodate}.

\begin{equation}
    P(\text{parent}_i) = \frac{\text{fitness}(\text{parent}_i)}{\sum_{j=1}^N \text{fitness}(\text{parent}_j)}
    \label{eq:tournament}
\end{equation}

\subsection{Cruce}
El operador de cruce (\textit{crossover}) tiene como propósito principal explorar nuevas regiones del espacio de búsqueda combinando las características favorables de dos soluciones ``padres" para generar nuevas soluciones ``hijas" potencialmente mejores. Algunas versiones realizan cruce siempre, mientras que otras proponen una probabilidad para que dos soluciones puedan cruzarse, de forma que ocurra más esporádicamente.

\textbf{One-point crossover (binario)}: Se elige al azar un punto en los cromosomas de
ambos progenitores y se designa como ``punto de cruce”. Los bits a la
derecha de ese punto se intercambian entre los dos cromosomas parentales.
El resultado son dos descendientes, cada uno con información genética de
ambos progenitores \cite{DAGDIA2020283}.
Dados dos padres (los cromosomas) \( P_1 = (p_1^1, p_2^1, \dots, p_n^1) \) y \( P_2 = (p_1^2, p_2^2, \dots, p_n^2) \), se selecciona un punto de cruce \( k \) aleatorio. Los hijos \( H_1 \) y \( H_2 \) se generan como:

\begin{equation}
H_1 = (p_1^1, p_2^1, \dots, p_k^1, p_{k+1}^2, \dots, p_n^2)
\label{eq:one_point_h1}
\end{equation}

\begin{equation}
H_2 = (p_1^2, p_2^2, \dots, p_k^2, p_{k+1}^1, \dots, p_n^1)
\label{eq:one_point_h2}
\end{equation}

\textbf{Blend crossover (continuo)}: Dado dos números reales para cada uno de los genes de
los padres al hijo se le asignará un número aleatorio entre ese rango de gen
para cada gen que conforme al vector cromosómico \cite{purduelecture}.
Dados dos padres \( P_1 = (p_1^1, p_2^1, \dots, p_n^1) \) y \( P_2 = (p_1^2, p_2^2, \dots, p_n^2) \), para cada gen \( i \), se calcula un intervalo \( [c_{\text{min}}, c_{\text{max}}] \), donde:

\begin{equation}
c_{\text{min}} = \text{min}(p_i^1, p_i^2) - \alpha \cdot d
\label{eq:blx_cmin}
\end{equation}

\begin{equation}
c_{\text{max}} = \text{max}(p_i^1, p_i^2) + \alpha \cdot d
\label{eq:blx_cmax}
\end{equation}

\begin{equation}
d = |p_i^1 - p_i^2|
\label{eq:blx_d}
\end{equation}

El valor del gen \( i \) en el hijo \( H \) se genera aleatoriamente dentro del intervalo \( [c_{\text{min}}, c_{\text{max}}] \):

\begin{equation}
H_i = \text{rand}(c_{\text{min}}, c_{\text{max}})
\label{eq:blx_hi}
\end{equation}

Por supuesto existen muchos más tipos de cruces tanto para versiones en dominio continuo como en dominio discreto y binario, pero los principales operadores y que más suelen utilizarse son los descritos.

\subsection{Mutación}
El operador de mutación tiene como propósito principal favorecer la exploración del espacio de búsqueda introduciendo perturbaciones aleatorias en las soluciones hijo. Este operador es más delicado que el de cruce y por lo general se propone un valor bajo de probabilidad de mutación, ya que demasiadas perturbaciones pueden afectar a los resultados. Un desajuste entre el ratio de exploración y explotación a favor de la exploración puede dar como resultado soluciones con un \textit{fitness} muy malo.

\textbf{Mutación binaria}: Este operador básico y clásico consiste en cambiar un \textit{bit} arbitrario de un genotipo o solución de un algoritmo genético binario a su estado inverso dada una probabilidad de mutación \cite{mirjalili2019genetic}.

\begin{equation}
x_i' = 
\begin{cases}
1 - x_i & \text{con probabilidad } p_m, \\
x_i     & \text{con probabilidad } 1 - p_m.
\end{cases}
\label{eq:binary_mutation}
\end{equation}

\textbf{Mutación continua}: Este tipo de mutación se utiliza especialmente en problemas en los que se busca una exploración más amplia del espacio de búsqueda, ya que la distribución de \textit{Cauchy} tiene colas pesadas y puede generar valores aleatorios alejados del centro con mayor probabilidad en comparación con una distribución normal. La mutación se define como:

\begin{equation}
m(x_i) = 
\begin{cases}
(2r)^{\frac{1}{\eta + 1}} - 1 & \text{si } p \leq 0.5, \\
1 - (2r)^{\frac{1}{\eta + 1}} & \text{si } p > 0.5,
\end{cases}
\label{eq:cauchy_mutation}
\end{equation}

donde:
\begin{itemize}
    \item \( r \) es un número aleatorio en el intervalo \([0, 1]\),
    \item \( \eta \) es una variable de control que ajusta la distribución,
    \item \( p \) es una probabilidad aleatoria en \([0, 1]\).
\end{itemize}

Esta función de mutación está diseñada para generar una exploración más amplia del espacio de búsqueda, con la posibilidad de generar valores más alejados del centro de la distribución.

Por supuesto y tal como se mencionó en el apartado de \textit{crossover}, existen multitud de propuestas en cuanto a operadores de mutación.

\section{Informe bibliométrico}

\begin{figure}[htp]
    \centering
    \includegraphics[width=1\linewidth]{img/scopus_chart_gas.png}
    \caption{Número de artículos desde $2010-2024$ relacionados con los algoritmos genéticos en \textit{Scopus}.}
    \label{fig:scopus}
\end{figure}

Como puede verse en la gráfica de la figura \ref{fig:scopus}, los algoritmos genéticos son, pese a su longeva existencia, cada vez más populares. El número de artículos por año crece y crece cada vez más sin ninguna tendencia evidente a bajar, lo que parece indicar que los algoritmos genéticos son más populares que nunca y se sigue innovando en este sector.

Estos datos han sido obtenido mediante el buscador de \textit{Scopus}. Utilizando una herramienta como \textit{OpenAlex} se han obtenido datos complementarios. Por ejemplo, se puede observar que la gran mayoría de las publicaciones son (fig \ref{fig:openalex-closed}) cerradas (se accede mediante pasarelas de pago o acceso institucional). Otros datos interesantes se pueden obtener de este portal, como cuáles son las instituciones más prolíficas (fig \ref{fig:openalex-institutions}) en relación a los \textbf{GAs} y cuáles son los tipos de publicaciones más comunes (fig \ref{fig:openalex-type}) (un $88\%$ son artículos).

\begin{figure}[htp]
    \centering
    \includegraphics[width=1\linewidth]{img/open-access.png}
    \caption{Publicaciones cerradas vs abiertas sobre \textbf{GAs}.}
    \label{fig:openalex-closed}
\end{figure}

\begin{figure}[htp]
    \centering
    \includegraphics[width=1\linewidth]{img/top_institutions.png}
    \caption{Instituciones que más publican sobre \textbf{GAs}.}
    \label{fig:openalex-institutions}
\end{figure}

\begin{figure}[htp]
    \centering
    \includegraphics[width=1\linewidth]{img/type.png}
    \caption{Tipo de publiaciones sobre las \textbf{GAs}.}
    \label{fig:openalex-type}
\end{figure}

A partir de los artículos más citados en Scopus relacionados con Genetic Algorithms (\textbf{GAs}), los temas más candentes parecen estar relacionados con:
\begin{enumerate}
    \item \textbf{Optimización y mejora de algoritmos genéticos}
    \begin{itemize}
        \item Revisión del estado del arte de los \textbf{GAs} y sus mejoras a lo largo del tiempo.
        \item Optimización de hiperparámetros en modelos de aprendizaje automático, clave para la implementación eficiente de \textbf{GAs}.
    \end{itemize}

    \item \textbf{Metaheurísticas bioinspiradas y nuevas variantes}
    \begin{itemize}
        \item Desarrollo de nuevos algoritmos de optimización como \textit{Marine Predators Algorithm}, \textit{Equilibrium Optimizer} y \textit{Arithmetic Optimization Algorithm}.
        \item Técnicas para mejorar la convergencia y evitar estancamientos en óptimos locales.
    \end{itemize}

    \item \textbf{Aplicaciones en biología y bioinformática}
    \begin{itemize}
        \item Uso de \textbf{GAs} para predecir estructuras de proteínas con aprendizaje profundo.
        \item Aplicaciones en la clasificación evolutiva de sistemas \textit{CRISPR-Cas}.
    \end{itemize}

    \item \textbf{Optimización multiobjetivo y herramientas avanzadas}
    \begin{itemize}
        \item Desarrollo de herramientas como \textit{Pymoo}, una biblioteca en Python para optimización multiobjetivo con \textbf{GAs} y otras metaheurísticas.
    \end{itemize}
\end{enumerate}

\section{Hibridaciones}
Los algoritmos híbridos entre genéticos y otras técnicas de búsqueda son muy comunes y ampliamente utilizados. Los algoritmos genéticos se combinan con otros métodos de optimización, como búsqueda local o enfriamiento simulado, con el objetivo de mejorar la eficiencia y precisión.

El objetivo detrás de este tipo de algoritmos es utilizar a las \textbf{GAs} como método principal de búsqueda y exploración. Tras ir poco a poco convergiendo a una solución, se suelen utilizar métodos de búsqueda local o enfriamiento para mejorar y refinar la fase de explotación del algoritmo.

A continuación se detallan algunos de estos métodos \cite{García-Martínez2018} de hibridación y se explicarán las ventajas de cada uno y algunos ejemplos dentro de estos métodos de hibridación.

\subsection{Hibridaciones colaborativas}
Estas estrategias implican la interacción entre algoritmos independientes que intercambian información durante su ejecución.
\subsubsection{Colaborativas \textit{teamwork}}
Varios algoritmos se ejecutan en paralelo, compartiendo información periódicamente.
\begin{enumerate}
    \item Ejemplo $1$: Los \textbf{GA} distribuidos (\textbf{DGAs}) dividen la población en subpoblaciones (islas), cada una procesada por un \textbf{GA} independiente. Las migraciones entre islas introducen diversidad. Por ejemplo, en \cite{843494}, se utilizan subpoblaciones con distintos parámetros (alta mutación para exploración, baja para explotación).
    \item Ejemplo $2$: En \cite{cosearch}, se combinan \textbf{GA}, búsqueda tabú y búsqueda local. Los \textbf{GA} generan soluciones en zonas no exploradas usando una memoria adaptativa que registra el historial de búsqueda de otros algoritmos.
\end{enumerate}

\subsubsection{Colaborativas \textit{relay}}
Los algoritmos se ejecutan secuencialmente en una canalización.
\begin{enumerate}
    \item Ejemplo $1$: Un \textbf{GA} global seguido de un \textbf{GA} local. El primero explora el espacio de búsqueda, y el segundo refina las mejores soluciones encontradas.
    \item Ejemplo $2$: Un \textbf{GA} se combina con el método Nelder-Mead: el \textbf{GA} realiza una exploración global, y Nelder-Mead mejora las soluciones prometedoras. El método de Nelder-Mead es un método numérico utilizado para encontrar el mínimo o el máximo de una función objetivo en un espacio multidimensional. Es un método de búsqueda directa (basado en la comparación de funciones) y suele aplicarse a problemas de optimización no lineal para los que pueden no conocerse las derivadas.
\end{enumerate}

\subsection{Hibridaciones integrativas}
Un algoritmo actúa como componente de otro, integrándose en su funcionamiento interno.

\subsubsection{Integrativas \textit{teamwork}}
Un metaheurística se incrusta dentro de otra como componente clave.
\begin{enumerate}
    \item Ejemplo $1$: Algoritmos meméticos \cite{memetics}, donde un \textbf{GA} se combina con búsqueda local. 
    \item Ejemplo $2$: Micro-GA ($\mu$\textbf{GA}) \cite{930311}, usado como operador de refinamiento. Un \textbf{GA} con población pequeña (ej. $5$ individuos) mejora soluciones específicas del \textbf{GA} principal, aprovechando su capacidad para seguir crestas en espacios complejos.
\end{enumerate}

\subsubsection{Integrativas \textit{relay}}
Un \textbf{GA} realiza funciones específicas dentro de otra metaheurística dominante.

\begin{enumerate}
    \item Ejemplo $1$: Un \textbf{GA} \textit{steady-state} genera soluciones candidatas para un \textit{simulated annealing} (\textbf{SA}). El SA decide aceptar o rechazar las soluciones, integrando exploración (\textbf{GA}) y explotación (\textbf{SA}).
    \item Ejemplo $2$: En \cite{microchc}, un $\mu$\textbf{CHC} (variante de \textbf{GA} con alta presión selectiva) actúa como operador de perturbación en una búsqueda local iterada, introduciendo diversidad sin perder calidad.
\end{enumerate}

Las hibridaciones de algoritmos genéticos representan un campo en constante evolución dentro de la computación evolutiva y la optimización metaheurística. El éxito demostrado por estas técnicas híbridas ha llevado a un creciente interés en desarrollar nuevas combinaciones y estrategias que puedan abordar problemas cada vez más complejos. La tendencia actual apunta hacia la integración con técnicas de aprendizaje profundo y otros métodos de inteligencia artificial.

\newpage
\section{Bibliografía}

\printbibliography[heading=none, category=cited]
\end{document}
