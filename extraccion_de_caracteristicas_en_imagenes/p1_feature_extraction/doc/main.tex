\documentclass[12pt,letterpaper]{article}
\usepackage[a4paper, top=1.2in, bottom=1.4in, left=1in, right=1in]{geometry}
\usepackage{graphicx} % Required for inserting images
\graphicspath{ {./img/} }
\usepackage[spanish]{babel}
\usepackage{float}
\usepackage{fancyhdr}
\setlength{\parskip}{1em}  % Adds space between paragraphs (1em)
\usepackage{amsmath,amssymb}
\usepackage{tikz}
\newcommand{\tikzmark}[1]{\tikz[baseline,remember picture] \coordinate (#1) {};}
\usetikzlibrary{positioning}
\usetikzlibrary{shadows,arrows.meta} % For adding edges label
\usetikzlibrary{calc}
\usepackage{eso-pic}
\usepackage[backend=biber, defernumbers=true, citestyle=numeric-comp, bibstyle=ieee, sorting=none]{biblatex}
\addbibresource{bibliography/bibliography.bib}
\DeclareBibliographyCategory{cited}
\AtEveryCitekey{\addtocategory{cited}{\thefield{entrykey}}}
% Configurando BibLaTeX
\DefineBibliographyStrings{spanish}{
  url = {URL},
  andothers={et ~al\adddot}
}
\usepackage{listings}
\usepackage{xcolor}

\definecolor{codegreen}{rgb}{0,0.6,0}
\definecolor{codegray}{rgb}{0.5,0.5,0.5}
\definecolor{codepurple}{rgb}{0.58,0,0.82}
\definecolor{backcolour}{rgb}{0.95,0.95,0.92}

\lstdefinestyle{mystyle}{
    backgroundcolor=\color{backcolour},   
    commentstyle=\color{codegreen},
    keywordstyle=\color{magenta},
    numberstyle=\tiny\color{codegray},
    stringstyle=\color{codepurple},
    basicstyle=\ttfamily\footnotesize,
    breakatwhitespace=false,         
    breaklines=true,                 
    captionpos=b,                    
    keepspaces=true,                 
    numbers=left,                    
    numbersep=5pt,                  
    showspaces=false,                
    showstringspaces=false,
    showtabs=false,                  
    tabsize=2
}

\lstset{style=mystyle}

\AddToShipoutPictureBG{%
\begin{tikzpicture}[remember picture, overlay]
\node[opacity=.15, inner sep=0pt]
    at(current page.center){\includegraphics[scale=1.5]{img/logo-ugr2.png}};
\end{tikzpicture}%
}

\title{Extracción de Características en Imágenes - Práctica 1}
\author{Miguel García López}
\date{Diciembre 2024}

\pagestyle{fancyplain}
\headheight 35pt
\lhead{Miguel García López}            
\chead{\textbf{\small Práctica 1}}
\rhead{Master Ciencia de Datos \\ \today}
\lfoot{\scriptsize\LaTeX}
\cfoot{\small Extracción de Características en Imágenes}
\rfoot{\small\thepage}
\headsep 1.5em

\author{Miguel García López} % Nombre y apellidos

\date{\normalsize\today} % Incluye la fecha actual

\begin{document}
\begin{titlepage}
\begin{figure}
    \vspace{-1.3cm}
    \begin{center}
        \includegraphics[width=0.75\textwidth]{img/UGR-Logo.png}
    \end{center}
\end{figure}
\vspace{1.3cm}
\centering
\normalfont \normalsize
\textsc{\textbf{Extracción de Características en Imágenes 2024-2025} \\ \vspace{.15cm} Master Ciencia de Datos\\ \vspace{.15cm} Universidad de Granada} \\ [25pt] 
    \huge Práctica 1

\normalfont \normalsize \vspace{.30cm}
\textsc{Miguel García López}

\end{titlepage}

\tableofcontents
\listoffigures
\listoftables
\newpage

\section{Introducción}
En la presente práctica de la asignatura de \textbf{Extracción de Características en Imágenes}, se llevarán a cabo una serie de tareas definidas por un grafo de decisión (fig \ref{fig:de_graph}). Dado este grafo es necesario seguir el camino hasta al final y allí donde haya una bifurcación, escoger entre una tarea básica o una tarea bonificadora. Las tareas bonificadores son iguales que las básicas, pero con un toque de dificultad y desarrollo por parte del alumno. De forma resumida, las tareas a realizar son las siguientes: 

\begin{itemize}
    \item \textbf{Búsqueda de un conjunto de datos:} El estudiante puede usar el \textit{dataset} \textbf{MNIST} por defecto, pero en este caso se ha optado por la tarea complementaria de escoger uno.
    \item \textbf{Clasificación con HOG:} Se entrenará un modelo \textbf{SVM} usando el descriptor \textbf{HOG} y se realizará un análisis de los resultados del mismo. Además se aplicarán técnicas como validación cruzada y selección de hiperparámetros.
    \item \textbf{Clasificación con LBP:} Se realizará lo mismo que con \textbf{HOG} descrito en el apartado anterior. Además, como parte de la tarea complementaria bonificada, se usará una implementación propia de \textbf{LBP} para extraer las características del \textit{dataset}.
\end{itemize}

De las $24000$ imágenes se han escogido de forma aleatoria, y teniendo en cuenta equilibrio entre clases, $8000$ imágenes para la clasificación con \textbf{SVM+LBP} y \textbf{SVM+HOG}.

\begin{figure}[htp]
    \centering
    \includegraphics[width=0.6\linewidth]{img/decision_graph}
    \caption{Grafo de decisión de tareas.}
    \label{fig:de_graph}
\end{figure}

\section{Dataset}
Para el \textit{dataset} se ha escogido el conjunto de ``Gatos vs Perros" de \textit{Kaggle}. Este se puede encontrar en el siguiente enlace: \textit{https://www.kaggle.com/competitions/dogs-vs-cats}.\\[6pt]
El conjunto contiene cerca de $24000$ imágenes, la mitad de perros y la mitad de gatos. Este conjunto se compone de imágenes de multitud de resoluciones, por lo que se ha procedido a realizar varias pruebas en el código y se ha llegado a la conclusión de que $28\times 28$ píxeles es un tamaño con el que poder trabajar por los siguiente motivos:
\begin{itemize}
    \item La extracción de descriptores en imágenes de alta resolución lleva a altos tiempos de cómputo.
    \item La implementación de \textbf{LBP} es rápida, pero no tanto como las implementaciones de otros descriptores como \textbf{HOG} en \textit{OpenCV}.
    \item Se han realizado varias pruebas y con tamaños de resolución mucho mayores no se consiguen unos resultados mucho mejores (hasta donde se ha podido comprobar).
\end{itemize}
Además de lo descrito, se han transformado las imágenes a escala de grises para trabajar con un solo canal. 

\section{Clasificación con HOG}
\subsection{HOG}
El descriptor \textbf{HOG} es una técnica ampliamente utilizada en la visión por computador, cuyo objetivo es capturar la estructura local de las imágenes basándose en los gradientes de intensidad. Divide la imagen (fig \ref{fig:hog}) en celdas pequeñas y calcula un histograma de orientaciones de gradiente dentro de cada celda. Para mejorar la robustez frente a cambios de iluminación, se normalizan los histogramas en bloques de celdas adyacentes.

\begin{figure}[htp]
    \centering
    \includegraphics[width=1\linewidth]{img/hog}
    \caption{Flujo de cómputo del descriptor HOG.}
    \label{fig:hog}
\end{figure}

\subsection{Búsqueda de hiperparámetros}
Se ha implementado una búsqueda de hiperparámetros de forma que es posible buscar solo hiperparámetros del algoritmos \textbf{SVM} o, si se selecciona, búsqueda para \textbf{SVM} y \textbf{HOG}. Hay que tener en cuenta que la búsqueda de hiperparámetros es un proceso muy costoso computacionalmente, por lo que esta búsqueda en lo relativo a \textbf{HOG} se ha realizado con $200$ imágenes usando el algoritmo \textbf{RandomSearch} de \textit{Scikit-Learn}. Las principales ventajas de este algoritmo son:
\begin{itemize}
    \item Un espacio de búsqueda más amplio al realizar combinaciones aleatorias.
    \item Más eficiente en espacios de dimensionalidad alta.
\end{itemize}
Los parámetros optimizados no han sido todos. Inicialmente se realizó un estudio de los parámetros tanto experimental como teórico. Dado ese primer paso se decidió que \texttt{winSize} sería del tamaño de la imagen, \texttt{blockSize} la mitad y \texttt{blockStride} y \texttt{cellSize} un cuarto. Se realizan búsquedas sobre los siguientes parámetros:

\begin{itemize}
    \item \texttt{nbins}: Número de histogramas por celda. Valores: $\{6, 9, 12\}$.
    \item \texttt{winSigma}: Sigma de la ventana de suavizado gaussiano. Valores: $\{0.5, 1.0, 2.0, 5.0\}$.
    \item \texttt{L2HysThreshold}: Umbral para la normalización L2. Valores: $\{0.1, 0.2, 0.3, 0.4\}$.
    \item \texttt{signedGradients}: Indicador de gradientes firmados (booleano).
    \item \texttt{gammaCorrection}: Aplicación de corrección gamma (booleano).
\end{itemize}

\begin{table}[htp]
    \centering
    \begin{tabular}{ll}
    \hline
    \textbf{Parámetro} & \textbf{Valor} \\
    \hline
    svm\_kernel & rbf \\
    svm\_gamma & 1 \\
    svm\_C & 1 \\
    \hline
    \end{tabular}
    \caption{Parámetros del modelo SVM.}
    \label{tab:svm_params}
\end{table}

\begin{table}[htp]
    \centering
    \begin{tabular}{ll}
    \hline
    \textbf{Parámetro} & \textbf{Valor} \\
    \hline
    winSize & (28, 28) \\
    blockSize & (14, 14) \\
    blockStride & (7, 7) \\
    cellSize & (7, 7) \\
    nbins & 12 \\
    winSigma & 5 \\
    L2HysThreshold & 0.3 \\
    signedGradients & True \\
    gammaCorrection & 0 \\
    \hline
    \end{tabular}
    \caption{Parámetros del descriptor HOG.}
    \label{tab:hog_params}
\end{table}

Los valores escogidos finalmente para los parámetros tanto del descriptor como para el \textbf{SVM} se encuentran descritos en las tablas \ref{tab:hog_params},\ref{tab:svm_params}.

\newpage
\section{Bibliografía}

\printbibliography[heading=none, category=cited]
\end{document}
