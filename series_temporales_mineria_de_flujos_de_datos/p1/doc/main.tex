\documentclass[12pt,letterpaper]{article}
\usepackage[a4paper, top=1.2in, bottom=1.4in, left=1in, right=1in]{geometry}
\usepackage{graphicx} % Required for inserting images
\graphicspath{ {./img/} }
\usepackage[spanish]{babel}
\usepackage{float}
\usepackage{fancyhdr}
\setlength{\parskip}{1em}  % Adds space between paragraphs (1em)
\usepackage{amsmath,amssymb}
\usepackage{booktabs}
\usepackage{tikz}
\usepackage{subcaption}
\newcommand{\tikzmark}[1]{\tikz[baseline,remember picture] \coordinate (#1) {};}
\usetikzlibrary{positioning}
\usetikzlibrary{shadows,arrows.meta} % For adding edges label
\usetikzlibrary{calc}
\usepackage{eso-pic}
\usepackage[backend=biber, defernumbers=true, citestyle=numeric-comp, bibstyle=ieee, sorting=none]{biblatex}
\addbibresource{bibliography/bibliography.bib}
\DeclareBibliographyCategory{cited}
\AtEveryCitekey{\addtocategory{cited}{\thefield{entrykey}}}
% Configurando BibLaTeX
\DefineBibliographyStrings{spanish}{
  url = {URL},
  andothers={et ~al\adddot}
}
\usepackage{listings}
\usepackage{xcolor}

\definecolor{codegreen}{rgb}{0,0.6,0}
\definecolor{codegray}{rgb}{0.5,0.5,0.5}
\definecolor{codepurple}{rgb}{0.58,0,0.82}
\definecolor{backcolour}{rgb}{0.95,0.95,0.92}

\lstdefinestyle{mystyle}{
    backgroundcolor=\color{backcolour},   
    commentstyle=\color{codegreen},
    keywordstyle=\color{magenta},
    numberstyle=\tiny\color{codegray},
    stringstyle=\color{codepurple},
    basicstyle=\ttfamily\footnotesize,
    breakatwhitespace=false,         
    breaklines=true,                 
    captionpos=b,                    
    keepspaces=true,                 
    numbers=left,                    
    numbersep=5pt,                  
    showspaces=false,                
    showstringspaces=false,
    showtabs=false,                  
    tabsize=2
}

\lstset{style=mystyle}

\AddToShipoutPictureBG{%
\begin{tikzpicture}[remember picture, overlay]
\node[opacity=.15, inner sep=0pt]
    at(current page.center){\includegraphics[scale=1.5]{img/logo-ugr2.png}};
\end{tikzpicture}%
}

\title{Práctica: Series Temporales}
\author{Miguel García López}
\date{Marzo 2025}

\pagestyle{fancyplain}
\headheight 35pt
\lhead{Miguel García López}            
\chead{\textbf{\small Práctica: Series Temporales}}
\rhead{Master Ciencia de Datos \\ \today}
\lfoot{\scriptsize\LaTeX}
\cfoot{\small Práctica: Series Temporales}
\rfoot{\small\thepage}
\headsep 1.5em

\author{Miguel García López} % Nombre y apellidos

\date{\normalsize\today} % Incluye la fecha actual

\begin{document}
\begin{titlepage}
    \begin{figure}
        \vspace{-1.3cm}
        \begin{center}
            \includegraphics[width=0.75\textwidth]{img/UGR-Logo.png}
        \end{center}
    \end{figure}
    \vspace{1.3cm}
    \centering
    \normalfont \normalsize
    \textsc{\textbf{Práctica: Series Temporales 2024-2025} \\ \vspace{.15cm} Master Ciencia de Datos\\ \vspace{.15cm} Universidad de Granada} \\ [25pt]
    \huge Práctica: Series Temporales

    \normalfont \normalsize \vspace{.30cm}
    \textsc{Miguel García López}

\end{titlepage}

\tableofcontents
\listoffigures
\listoftables
\newpage

\section{Introducción}
La práctica de la asignatura de \textbf{Series Temporales y Minería de Flujos de Datos} consiste en, a partir de una conjunto de datos dado, revolser un problema de series temporales. Concretamente el conjunto de datos dado es \textbf{Oikolab Weather}, el cual contiene $8$ series temporales sobre datos climáticos, aunque en esta práctica se debe trabajar tan solo con la variable de temperatura. Los datos se empezaron a medir el $1$ de Enero de $2010$.

El objetivo es predecir la temperatura para los meses restantes del año donde termina la serie, es decir, $2021$. Según se avance en el trabajo, se irán respondiendo cuestiones planteadas en el guión de prácticas.

\section{Tareas}
\subsection{¿Es necesario realizar algún tipo de preprocesamiento en la serie? Tanto en el caso
afirmativo como en el negativo, justifique su respuesta e incluya el código Python
requerido, si es el caso.}

Considero que no es necesario aplicar ningún tipo de preprocesamiento a la serie temporal. Analizando los datos no se ha encontrado ningún valor faltante. La serie no parece tener ningún tipo de valor anómalo a simple vista y parece bastante sencilla.

Además, aunque se vaya a responder en las siguientes cuestiones, la serie parece estacionaria, es decir, no presenta tendencia y su varianza es constante a lo largo de los meses, por lo que no aplicaría ninguna transformación (como la logarítimica).

\subsection{¿Tiene tendencia la serie? Tanto en el caso afirmativo como en el negativo,
justifique su respuesta e incluya el código Python requerido, si es el caso,
justificando el modelo de tendencia.}

La serie no presenta tendencia ya que la media de esta es constante. Esto se puede analizar visualmente a través de la imagen en la figura \ref{fig:temperature_ts}.

\begin{figure}[htp]
    \centering
    \includegraphics[width=0.3\linewidth]{../../temperature_train.jpg}
    \caption{Serie temporal de la temperatura (partición de \textit{train}).}
    \label{fig:temperature_ts}
\end{figure}

\end{document}
