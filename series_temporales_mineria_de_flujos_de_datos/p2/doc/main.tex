\documentclass[12pt,letterpaper]{article}
\usepackage[a4paper, top=1.2in, bottom=1.4in, left=1in, right=1in]{geometry}
\usepackage{graphicx} % Required for inserting images
\graphicspath{ {./img/} }
\usepackage[spanish]{babel}
\usepackage{float}
\usepackage{fancyhdr}
\setlength{\parskip}{1em}  % Adds space between paragraphs (1em)
\usepackage{amsmath,amssymb}
\usepackage{booktabs}
\usepackage{tikz}
\usepackage{listings}
\usepackage[utf8]{inputenc}

\usepackage{xcolor}

\lstset{
    language=Python,
    basicstyle=\ttfamily\footnotesize,
    keywordstyle=\color{blue}\bfseries,
    commentstyle=\color{gray}\itshape,
    stringstyle=\color{orange},
    showstringspaces=false,
    frame=single,
    breaklines=true
}
\usepackage{subcaption}
\newcommand{\tikzmark}[1]{\tikz[baseline,remember picture] \coordinate (#1) {};}
\usetikzlibrary{positioning}
\usetikzlibrary{shadows,arrows.meta} % For adding edges label
\usetikzlibrary{calc}
\usepackage{eso-pic}
\usepackage[backend=biber, defernumbers=true, citestyle=numeric-comp, bibstyle=ieee, sorting=none]{biblatex}
\addbibresource{bibliography/bibliography.bib}
\DeclareBibliographyCategory{cited}
\AtEveryCitekey{\addtocategory{cited}{\thefield{entrykey}}}
% Configurando BibLaTeX
\DefineBibliographyStrings{spanish}{
  url = {URL},
  andothers={et ~al\adddot}
}
\usepackage{listings}
\usepackage{xcolor}

\definecolor{codegreen}{rgb}{0,0.6,0}
\definecolor{codegray}{rgb}{0.5,0.5,0.5}
\definecolor{codepurple}{rgb}{0.58,0,0.82}
\definecolor{backcolour}{rgb}{0.95,0.95,0.92}

\lstdefinestyle{mystyle}{
    backgroundcolor=\color{backcolour},   
    commentstyle=\color{codegreen},
    keywordstyle=\color{magenta},
    numberstyle=\tiny\color{codegray},
    stringstyle=\color{codepurple},
    basicstyle=\ttfamily\footnotesize,
    breakatwhitespace=false,         
    breaklines=true,                 
    captionpos=b,                    
    keepspaces=true,                 
    numbers=left,                    
    numbersep=5pt,                  
    showspaces=false,                
    showstringspaces=false,
    showtabs=false,                  
    tabsize=2
}

\lstset{style=mystyle}

\AddToShipoutPictureBG{%
\begin{tikzpicture}[remember picture, overlay]
\node[opacity=.15, inner sep=0pt]
    at(current page.center){\includegraphics[scale=1.5]{img/logo-ugr2.png}};
\end{tikzpicture}%
}

\title{Práctica: Minería de Flujos de Datos}
\author{Miguel García López}
\date{Marzo 2025}

\pagestyle{fancyplain}
\headheight 35pt
\lhead{Miguel García López}            
\chead{\textbf{\small Práctica: Minería de Flujos de Datos}}
\rhead{Master Ciencia de Datos \\ \today}
\lfoot{\scriptsize\LaTeX}
\cfoot{\small Práctica: Minería de Flujos de Datos}
\rfoot{\small\thepage}
\headsep 1.5em

\author{Miguel García López} % Nombre y apellidos

\date{\normalsize\today} % Incluye la fecha actual

\begin{document}
\begin{titlepage}
    \begin{figure}
        \vspace{-1.3cm}
        \begin{center}
            \includegraphics[width=0.75\textwidth]{img/UGR-Logo.png}
        \end{center}
    \end{figure}
    \vspace{1.3cm}
    \centering
    \normalfont \normalsize
    \textsc{\textbf{Práctica: Minería de Flujos de Datos 2024-2025} \\ \vspace{.15cm} Master Ciencia de Datos\\ \vspace{.15cm} Universidad de Granada} \\ [25pt]
    \huge Práctica: Minería de Flujos de Datos
    \normalfont \normalsize \vspace{.30cm}
    \textsc{Miguel García López}

\end{titlepage}

\tableofcontents
\listoffigures
\listoftables
\newpage

\section{Cuestiones}
\subsection{Explica en qué consisten los diferentes modos de evaluación/validación para clasificación en flujos de datos}

En clasificación, concretamente en entornos de flujo de datos, los métodos de evaluación difieren de los enfoques estáticos tradicionales debido a la naturaleza dinámica, infinita y potencialmente no estacionaria de los flujos.

\textbf{Holdout}
Se toman instantáneas en diferentes momentos durante el entrenamiento del modelo para ver cómo varía la métrica de calidad. Sólo es válido si el conjunto de \textit{test} es similar a los datos actuales (sin \textit{concept drift}) \cite{Casillas2025}.

\textbf{Test-Then-Train}
Este enfoque procesa cada nuevo dato en dos fases secuenciales: primero evalúa el modelo (\textit{test}) y luego lo actualiza (\textit{train}). Simula entornos reales de flujos continuos y proporciona métricas en tiempo real, como precisión acumulada.

\textbf{Prequencial}
Variante de \textit{Test-Then-Train} que calcula métricas en ventanas deslizantes o bloques. Utiliza dos enfoques: ventanas fijas (evalúa últimos nn datos, como $1000$ ejemplos) o ventanas adaptativas (ajusta el tamaño según detección de \textit{drift}, como el algoritmo \textit{ADWIN} \cite{Bifet2007}). Su ventaja principal es reducir el sesgo hacia datos antiguos. En \cite{Gama2010}, se aplicó en flujos financieros para medir la adaptabilidad de modelos ante cambios de mercado.

\textbf{Interleaved Validation}
Adaptación de la validación cruzada tradicional: divide el flujo en bloques temporales y los rota para entrenamiento y prueba. Este método es útil para evaluar robustez frente a \textit{drift}. En \cite{Gama2014}, se empleó para comparar algoritmos como \textit{VFDT} y \textit{Hoeffding Adaptive Tree} en presencia de cambios sintéticos en la distribución.

\textbf{Ventanas Deslizantes (Sliding Windows)}
Evalúa el modelo solo en datos recientes. Las ventanas pueden ser fijas (mantienen tamaño constante, como los últimos $10000$ ejemplos) o adaptativas (ajustan dinámicamente el tamaño usando umbrales de error \cite{Bifet2009}). Un ejemplo clásico es \textit{VFDT} \cite{Domingos2000}, que usa ventanas para limitar el uso de memoria en flujos infinitos, descartando datos obsoletos.

\subsection{Describe tres algoritmos de clasificación en flujos de datos y compara
ventajas/desventajas}
El primer algoritmo y uno de los más usados es el \textbf{VFDT} (\textit{Very Fast Decision Tree}) o Árbol de \textit{Hoeffding}.
El \textbf{VFDT}, propuesto por \textit{Domingos} y \textit{Hulten} ($2000$), es un algoritmo incremental que construye árboles de decisión utilizando el \textit{Hoeffding bound} (HB), un límite estadístico que garantiza con alta probabilidad que la mejor división en un nodo, basada en una muestra de datos, será la misma que si se usara el flujo completo. Opera en tiempo constante por muestra y memoria limitada. La cota \textit{Hoeffding} se describe como:
$$HB=\sqrt\frac{R^2ln(1/\delta)}{2n}$$
Donde $R$ es el rango de clases (diferencia máxima posible en las métricas de división, como ganancia de información), $n$ el número de muestras en el nodo, y $\delta$ la probabilidad de error \cite{Domingos2000}.

Usa la cota de \textit{Hoeffding} para garantizar que, con alta probabilidad, la mejor elección de división con una muestra será la misma que si se usara todo el flujo de datos. Esto permite hacer divisiones rápidamente sin necesidad de almacenar todos los datos históricos.



\newpage
\section{Bibliografía}

\printbibliography[heading=none, category=cited]
\end{document}
